% Generated by Sphinx.
\def\sphinxdocclass{report}
\documentclass[letterpaper,10pt,english]{sphinxmanual}
\usepackage[utf8]{inputenc}
\DeclareUnicodeCharacter{00A0}{\nobreakspace}
\usepackage[T1]{fontenc}
\usepackage{babel}
\usepackage{times}
\usepackage[Bjarne]{fncychap}
\usepackage{longtable}
\usepackage{sphinx}


\title{Sator Documentation}
\date{January 29, 2012}
\release{.01}
\author{Caleb Smith}
\newcommand{\sphinxlogo}{}
\renewcommand{\releasename}{Release}
\makeindex

\makeatletter
\def\PYG@reset{\let\PYG@it=\relax \let\PYG@bf=\relax%
    \let\PYG@ul=\relax \let\PYG@tc=\relax%
    \let\PYG@bc=\relax \let\PYG@ff=\relax}
\def\PYG@tok#1{\csname PYG@tok@#1\endcsname}
\def\PYG@toks#1+{\ifx\relax#1\empty\else%
    \PYG@tok{#1}\expandafter\PYG@toks\fi}
\def\PYG@do#1{\PYG@bc{\PYG@tc{\PYG@ul{%
    \PYG@it{\PYG@bf{\PYG@ff{#1}}}}}}}
\def\PYG#1#2{\PYG@reset\PYG@toks#1+\relax+\PYG@do{#2}}

\def\PYG@tok@gd{\def\PYG@tc##1{\textcolor[rgb]{0.63,0.00,0.00}{##1}}}
\def\PYG@tok@gu{\let\PYG@bf=\textbf\def\PYG@tc##1{\textcolor[rgb]{0.50,0.00,0.50}{##1}}}
\def\PYG@tok@gt{\def\PYG@tc##1{\textcolor[rgb]{0.00,0.25,0.82}{##1}}}
\def\PYG@tok@gs{\let\PYG@bf=\textbf}
\def\PYG@tok@gr{\def\PYG@tc##1{\textcolor[rgb]{1.00,0.00,0.00}{##1}}}
\def\PYG@tok@cm{\let\PYG@it=\textit\def\PYG@tc##1{\textcolor[rgb]{0.25,0.50,0.56}{##1}}}
\def\PYG@tok@vg{\def\PYG@tc##1{\textcolor[rgb]{0.73,0.38,0.84}{##1}}}
\def\PYG@tok@m{\def\PYG@tc##1{\textcolor[rgb]{0.13,0.50,0.31}{##1}}}
\def\PYG@tok@mh{\def\PYG@tc##1{\textcolor[rgb]{0.13,0.50,0.31}{##1}}}
\def\PYG@tok@cs{\def\PYG@tc##1{\textcolor[rgb]{0.25,0.50,0.56}{##1}}\def\PYG@bc##1{\colorbox[rgb]{1.00,0.94,0.94}{##1}}}
\def\PYG@tok@ge{\let\PYG@it=\textit}
\def\PYG@tok@vc{\def\PYG@tc##1{\textcolor[rgb]{0.73,0.38,0.84}{##1}}}
\def\PYG@tok@il{\def\PYG@tc##1{\textcolor[rgb]{0.13,0.50,0.31}{##1}}}
\def\PYG@tok@go{\def\PYG@tc##1{\textcolor[rgb]{0.19,0.19,0.19}{##1}}}
\def\PYG@tok@cp{\def\PYG@tc##1{\textcolor[rgb]{0.00,0.44,0.13}{##1}}}
\def\PYG@tok@gi{\def\PYG@tc##1{\textcolor[rgb]{0.00,0.63,0.00}{##1}}}
\def\PYG@tok@gh{\let\PYG@bf=\textbf\def\PYG@tc##1{\textcolor[rgb]{0.00,0.00,0.50}{##1}}}
\def\PYG@tok@ni{\let\PYG@bf=\textbf\def\PYG@tc##1{\textcolor[rgb]{0.84,0.33,0.22}{##1}}}
\def\PYG@tok@nl{\let\PYG@bf=\textbf\def\PYG@tc##1{\textcolor[rgb]{0.00,0.13,0.44}{##1}}}
\def\PYG@tok@nn{\let\PYG@bf=\textbf\def\PYG@tc##1{\textcolor[rgb]{0.05,0.52,0.71}{##1}}}
\def\PYG@tok@no{\def\PYG@tc##1{\textcolor[rgb]{0.38,0.68,0.84}{##1}}}
\def\PYG@tok@na{\def\PYG@tc##1{\textcolor[rgb]{0.25,0.44,0.63}{##1}}}
\def\PYG@tok@nb{\def\PYG@tc##1{\textcolor[rgb]{0.00,0.44,0.13}{##1}}}
\def\PYG@tok@nc{\let\PYG@bf=\textbf\def\PYG@tc##1{\textcolor[rgb]{0.05,0.52,0.71}{##1}}}
\def\PYG@tok@nd{\let\PYG@bf=\textbf\def\PYG@tc##1{\textcolor[rgb]{0.33,0.33,0.33}{##1}}}
\def\PYG@tok@ne{\def\PYG@tc##1{\textcolor[rgb]{0.00,0.44,0.13}{##1}}}
\def\PYG@tok@nf{\def\PYG@tc##1{\textcolor[rgb]{0.02,0.16,0.49}{##1}}}
\def\PYG@tok@si{\let\PYG@it=\textit\def\PYG@tc##1{\textcolor[rgb]{0.44,0.63,0.82}{##1}}}
\def\PYG@tok@s2{\def\PYG@tc##1{\textcolor[rgb]{0.25,0.44,0.63}{##1}}}
\def\PYG@tok@vi{\def\PYG@tc##1{\textcolor[rgb]{0.73,0.38,0.84}{##1}}}
\def\PYG@tok@nt{\let\PYG@bf=\textbf\def\PYG@tc##1{\textcolor[rgb]{0.02,0.16,0.45}{##1}}}
\def\PYG@tok@nv{\def\PYG@tc##1{\textcolor[rgb]{0.73,0.38,0.84}{##1}}}
\def\PYG@tok@s1{\def\PYG@tc##1{\textcolor[rgb]{0.25,0.44,0.63}{##1}}}
\def\PYG@tok@gp{\let\PYG@bf=\textbf\def\PYG@tc##1{\textcolor[rgb]{0.78,0.36,0.04}{##1}}}
\def\PYG@tok@sh{\def\PYG@tc##1{\textcolor[rgb]{0.25,0.44,0.63}{##1}}}
\def\PYG@tok@ow{\let\PYG@bf=\textbf\def\PYG@tc##1{\textcolor[rgb]{0.00,0.44,0.13}{##1}}}
\def\PYG@tok@sx{\def\PYG@tc##1{\textcolor[rgb]{0.78,0.36,0.04}{##1}}}
\def\PYG@tok@bp{\def\PYG@tc##1{\textcolor[rgb]{0.00,0.44,0.13}{##1}}}
\def\PYG@tok@c1{\let\PYG@it=\textit\def\PYG@tc##1{\textcolor[rgb]{0.25,0.50,0.56}{##1}}}
\def\PYG@tok@kc{\let\PYG@bf=\textbf\def\PYG@tc##1{\textcolor[rgb]{0.00,0.44,0.13}{##1}}}
\def\PYG@tok@c{\let\PYG@it=\textit\def\PYG@tc##1{\textcolor[rgb]{0.25,0.50,0.56}{##1}}}
\def\PYG@tok@mf{\def\PYG@tc##1{\textcolor[rgb]{0.13,0.50,0.31}{##1}}}
\def\PYG@tok@err{\def\PYG@bc##1{\fcolorbox[rgb]{1.00,0.00,0.00}{1,1,1}{##1}}}
\def\PYG@tok@kd{\let\PYG@bf=\textbf\def\PYG@tc##1{\textcolor[rgb]{0.00,0.44,0.13}{##1}}}
\def\PYG@tok@ss{\def\PYG@tc##1{\textcolor[rgb]{0.32,0.47,0.09}{##1}}}
\def\PYG@tok@sr{\def\PYG@tc##1{\textcolor[rgb]{0.14,0.33,0.53}{##1}}}
\def\PYG@tok@mo{\def\PYG@tc##1{\textcolor[rgb]{0.13,0.50,0.31}{##1}}}
\def\PYG@tok@mi{\def\PYG@tc##1{\textcolor[rgb]{0.13,0.50,0.31}{##1}}}
\def\PYG@tok@kn{\let\PYG@bf=\textbf\def\PYG@tc##1{\textcolor[rgb]{0.00,0.44,0.13}{##1}}}
\def\PYG@tok@o{\def\PYG@tc##1{\textcolor[rgb]{0.40,0.40,0.40}{##1}}}
\def\PYG@tok@kr{\let\PYG@bf=\textbf\def\PYG@tc##1{\textcolor[rgb]{0.00,0.44,0.13}{##1}}}
\def\PYG@tok@s{\def\PYG@tc##1{\textcolor[rgb]{0.25,0.44,0.63}{##1}}}
\def\PYG@tok@kp{\def\PYG@tc##1{\textcolor[rgb]{0.00,0.44,0.13}{##1}}}
\def\PYG@tok@w{\def\PYG@tc##1{\textcolor[rgb]{0.73,0.73,0.73}{##1}}}
\def\PYG@tok@kt{\def\PYG@tc##1{\textcolor[rgb]{0.56,0.13,0.00}{##1}}}
\def\PYG@tok@sc{\def\PYG@tc##1{\textcolor[rgb]{0.25,0.44,0.63}{##1}}}
\def\PYG@tok@sb{\def\PYG@tc##1{\textcolor[rgb]{0.25,0.44,0.63}{##1}}}
\def\PYG@tok@k{\let\PYG@bf=\textbf\def\PYG@tc##1{\textcolor[rgb]{0.00,0.44,0.13}{##1}}}
\def\PYG@tok@se{\let\PYG@bf=\textbf\def\PYG@tc##1{\textcolor[rgb]{0.25,0.44,0.63}{##1}}}
\def\PYG@tok@sd{\let\PYG@it=\textit\def\PYG@tc##1{\textcolor[rgb]{0.25,0.44,0.63}{##1}}}

\def\PYGZbs{\char`\\}
\def\PYGZus{\char`\_}
\def\PYGZob{\char`\{}
\def\PYGZcb{\char`\}}
\def\PYGZca{\char`\^}
\def\PYGZsh{\char`\#}
\def\PYGZpc{\char`\%}
\def\PYGZdl{\char`\$}
\def\PYGZti{\char`\~}
% for compatibility with earlier versions
\def\PYGZat{@}
\def\PYGZlb{[}
\def\PYGZrb{]}
\makeatother

\begin{document}

\maketitle
\tableofcontents
\phantomsection\label{index::doc}


A python module for atonal music analysis.


\chapter{Features}
\label{index:features}\label{index:sator}\begin{itemize}
\item {} 
Create tone rows, pitch class sets and pitch sets.

\item {} \begin{description}
\item[{Each pitch or pitch class set can have its own properties including:}] \leavevmode\begin{itemize}
\item {} 
Ordered vs. Unordered

\item {} 
Multiset vs. Unique element sets

\item {} 
A configurable Modulus

\item {} 
Definable canonical operators - (TTO's used to determine SC membership)

\end{itemize}

\end{description}

\item {} 
Easily construct matrices, find prime forms and interval class vectors.

\item {} 
Compare pitch or pitch class sets with various similarity relations

\item {} 
Explore non-standard pitch class spaces.

\end{itemize}


\chapter{Requirements}
\label{index:requirements}\begin{itemize}
\item {} 
There are no requirements for sator.

\end{itemize}


\chapter{Installation}
\label{index:installation}
sator is available on \href{http://pypi.python.org/pypi/sator}{PyPI}, so the easiest way to install it is to use \href{http://pip.openplans.org/}{pip}:

\begin{Verbatim}[commandchars=@\[\]]
pip install sator
\end{Verbatim}


\chapter{Authors}
\label{index:authors}
Caleb Smith


\section{Contents}
\label{index:toc}\label{index:contents}

\subsection{Overview}
\label{_templates/overview:overview}\label{_templates/overview::doc}\label{_templates/overview:id1}

\subsubsection{Constructing Tone Rows and Pitch/Pitch Class Sets}
\label{_templates/overview:constructing-tone-rows-and-pitch-pitch-class-sets}
The core module in sator contains the three essential classes for instantiating
and manipulating tone rows, pitch sets, and pitch class sets.

To construct rows and sets, import the ToneRow, PCSet, and PSet classes from
the core module as follows:

\begin{Verbatim}[commandchars=\\\{\}]
\PYG{g+gp}{\textgreater{}\textgreater{}\textgreater{} }\PYG{k+kn}{from} \PYG{n+nn}{sator.core} \PYG{k+kn}{import} \PYG{n}{ToneRow}\PYG{p}{,} \PYG{n}{PCSet}\PYG{p}{,} \PYG{n}{PSet}
\end{Verbatim}

To instantiate an empty pitch set, or pitch class set, use:
\textgreater{}\textgreater{}\textgreater{} a = PSet()
\textgreater{}\textgreater{}\textgreater{} b = PCSet()
\begin{itemize}
\item {} 
ToneRow objects are excluded from the above example because by definition, they cannot be empty.

\end{itemize}

The classes' constructors take an optional number of positional arguments as pitches or pc's.
These arguments can be integers, lists, tuples, or another tone row, pitch class set, or pitch set object.
Any of the following are both valid and equivalent:
\textgreater{}\textgreater{}\textgreater{} a = PSet({[}0, 2, 4, 6, 8{]})
\textgreater{}\textgreater{}\textgreater{} a = PSet(0, 2, 4, 6, 8)
\textgreater{}\textgreater{}\textgreater{} a = PSet(0, {[}2, 4{]}, 6, 8)
\textgreater{}\textgreater{}\textgreater{} b = PCSet(a)

When constructing a pitch set from a pitch class set and converting back to a pitch set, care must be taken so that the the pitch data remains unmolested.
In this example, pitch set c may not contain the intended pitches from a:
\textgreater{}\textgreater{}\textgreater{} a = PSet(0, 25, -1)
\textgreater{}\textgreater{}\textgreater{} b = PCSet(a)
\textgreater{}\textgreater{}\textgreater{} c = PSet(b)
\textgreater{}\textgreater{}\textgreater{} print c
\textgreater{}\textgreater{}\textgreater{} Out: {[}0, 1, 11{]}

To maintain pitch data, use the .pitches property explicitly:
\textgreater{}\textgreater{}\textgreater{} c = PSet(a.pitches)
\textgreater{}\textgreater{}\textgreater{} print c
\textgreater{}\textgreater{}\textgreater{} Out: {[}0, 25, -1{]}

To understand this distinction, the next topic of discussion is \emph{data\_inspection}


\subsection{Core module}
\label{_templates/core::doc}\label{_templates/core:core-module}\label{_templates/core:module-core}\index{core (module)}\index{PCBase (class in core)}

\begin{fulllineitems}
\phantomsection\label{_templates/core:core.PCBase}\pysigline{\strong{class }\code{core.}\bfcode{PCBase}}{}
Base class for Tone rows and PC sets
\index{m() (core.PCBase method)}

\begin{fulllineitems}
\phantomsection\label{_templates/core:core.PCBase.m}\pysiglinewithargsret{\bfcode{m}}{\emph{sub\_n=0}}{}
Perform M on the object in place. If an argument is provided, also
transpose the object in place by that amount.

\end{fulllineitems}

\index{mi() (core.PCBase method)}

\begin{fulllineitems}
\phantomsection\label{_templates/core:core.PCBase.mi}\pysiglinewithargsret{\bfcode{mi}}{\emph{sub\_n=0}}{}
Perform M and I on the object in place. If an argument is provided,
also transpose the object in play by that amount.

\end{fulllineitems}

\index{t\_m() (core.PCBase method)}

\begin{fulllineitems}
\phantomsection\label{_templates/core:core.PCBase.t_m}\pysiglinewithargsret{\bfcode{t\_m}}{\emph{sub\_n=0}, \emph{sub\_m=0}}{}
Perform TnMm on the object in place, where n and m are positional
arguments. If n is not provided, it defaults to 0. If m is not provided
it defaults to the default\_m of the object.

\end{fulllineitems}


\end{fulllineitems}

\index{PCSet (class in core)}

\begin{fulllineitems}
\phantomsection\label{_templates/core:core.PCSet}\pysiglinewithargsret{\strong{class }\code{core.}\bfcode{PCSet}}{\emph{*args}, \emph{**kwargs}}{}
A Class for pitch class sets which adds pitch class only methods
\index{c() (core.PCSet method)}

\begin{fulllineitems}
\phantomsection\label{_templates/core:core.PCSet.c}\pysiglinewithargsret{\bfcode{c}}{}{}
Change the given object in place to its literal compliment.

\end{fulllineitems}

\index{z() (core.PCSet method)}

\begin{fulllineitems}
\phantomsection\label{_templates/core:core.PCSet.z}\pysiglinewithargsret{\bfcode{z}}{}{}
Change the given object in place to its Z-partner if possible.
Otherwise leave the object unchanged.

\end{fulllineitems}


\end{fulllineitems}

\index{PPCSetBase (class in core)}

\begin{fulllineitems}
\phantomsection\label{_templates/core:core.PPCSetBase}\pysiglinewithargsret{\strong{class }\code{core.}\bfcode{PPCSetBase}}{\emph{*args}, \emph{**kwargs}}{}
Base class for PCSet and PSet
\index{abstract\_compliment (core.PPCSetBase attribute)}

\begin{fulllineitems}
\phantomsection\label{_templates/core:core.PPCSetBase.abstract_compliment}\pysigline{\bfcode{abstract\_compliment}}{}
Returns a PCSet of the abstract compliment of the given object.

\end{fulllineitems}

\index{canon() (core.PPCSetBase method)}

\begin{fulllineitems}
\phantomsection\label{_templates/core:core.PPCSetBase.canon}\pysiglinewithargsret{\bfcode{canon}}{\emph{t}, \emph{i}, \emph{m}}{}
Takes arguments in the form of (T, I, M) where each is a boolean.
These arguments determine which TTO's are canonical. These TTO's are
used to determine an object's set-class.
(The default canonical operators are T and I, hence the common name
Tn/TnI type).
Ex:
\begin{quote}

a.canon(True, False, False)

a.prime would now give the Tn-type, and ignore inversion as an
operation for determining set-class membership.
\end{quote}

\end{fulllineitems}

\index{cardinality (core.PPCSetBase attribute)}

\begin{fulllineitems}
\phantomsection\label{_templates/core:core.PPCSetBase.cardinality}\pysigline{\bfcode{cardinality}}{}
Returns the cardinality of the given object.

\end{fulllineitems}

\index{clear() (core.PPCSetBase method)}

\begin{fulllineitems}
\phantomsection\label{_templates/core:core.PPCSetBase.clear}\pysiglinewithargsret{\bfcode{clear}}{}{}
Remove all pitches/pitch classes from the object.

\end{fulllineitems}

\index{each\_card() (core.PPCSetBase method)}

\begin{fulllineitems}
\phantomsection\label{_templates/core:core.PPCSetBase.each_card}\pysiglinewithargsret{\bfcode{each\_card}}{}{}
Yields every set with the same cardinality as the given object, taking
into account the object's modulus.

\end{fulllineitems}

\index{each\_prime() (core.PPCSetBase method)}

\begin{fulllineitems}
\phantomsection\label{_templates/core:core.PPCSetBase.each_prime}\pysiglinewithargsret{\bfcode{each\_prime}}{}{}
Yields each unique set-class in the modulus of the given object.

\end{fulllineitems}

\index{each\_set() (core.PPCSetBase method)}

\begin{fulllineitems}
\phantomsection\label{_templates/core:core.PPCSetBase.each_set}\pysiglinewithargsret{\bfcode{each\_set}}{}{}
Yields every possible set in the modulus of the given object.

\end{fulllineitems}

\index{forte (core.PPCSetBase attribute)}

\begin{fulllineitems}
\phantomsection\label{_templates/core:core.PPCSetBase.forte}\pysigline{\bfcode{forte}}{}
Returns the Forte name for the given object.

\end{fulllineitems}

\index{forte\_name() (core.PPCSetBase static method)}

\begin{fulllineitems}
\phantomsection\label{_templates/core:core.PPCSetBase.forte_name}\pysiglinewithargsret{\strong{static }\bfcode{forte\_name}}{\emph{fname}}{}
A static method that returns a PCSet object with the fort-name provided
as a string argument.
Returns an empty PCSet if the argument is not a string with a valid
Forte name.

\end{fulllineitems}

\index{fromint() (core.PPCSetBase static method)}

\begin{fulllineitems}
\phantomsection\label{_templates/core:core.PPCSetBase.fromint}\pysiglinewithargsret{\strong{static }\bfcode{fromint}}{\emph{integer}, \emph{modulus=12}}{}
Static method that returns a PCSet object with pc's generated from
their integer representation.
\begin{quote}
\begin{description}
\item[{Ex:}] \leavevmode
0 = {[}{]}, 1 = {[}0{]}, 2 = {[}1{]}, 3 = {[}0, 1{]}, 4 = {[}2{]}, 5 = {[}0, 2{]}
PCSet.fromint(5) returns PCSet({[}0, 2{]})

\end{description}
\end{quote}

\end{fulllineitems}

\index{get\_canon (core.PPCSetBase attribute)}

\begin{fulllineitems}
\phantomsection\label{_templates/core:core.PPCSetBase.get_canon}\pysigline{\bfcode{get\_canon}}{}
Returns a three tuple showing which TTO's are canonical for the given
object. These are in the order (T, I, M). Refer to canon() for details
on how these settings are used.

\end{fulllineitems}

\index{icv (core.PPCSetBase attribute)}

\begin{fulllineitems}
\phantomsection\label{_templates/core:core.PPCSetBase.icv}\pysigline{\bfcode{icv}}{}
Returns the interval class vector of the given object.

\end{fulllineitems}

\index{insert() (core.PPCSetBase method)}

\begin{fulllineitems}
\phantomsection\label{_templates/core:core.PPCSetBase.insert}\pysiglinewithargsret{\bfcode{insert}}{\emph{place}, \emph{pitch}}{}
Given arguments (place, pitch) insert the pitch at the place position.
Take care to inspect the object's pitches attribute rather than it's
\_\_repr\_\_, which uses the ppc attribute and may truncate duplicates.
If the position is too great, the pitch will be appended at the end.

\end{fulllineitems}

\index{invariance\_vector (core.PPCSetBase attribute)}

\begin{fulllineitems}
\phantomsection\label{_templates/core:core.PPCSetBase.invariance_vector}\pysigline{\bfcode{invariance\_vector}}{}
A property that returns the list of (n, m) pairs that produce an
invariant set via TnMm

\end{fulllineitems}

\index{literal\_compliment (core.PPCSetBase attribute)}

\begin{fulllineitems}
\phantomsection\label{_templates/core:core.PPCSetBase.literal_compliment}\pysigline{\bfcode{literal\_compliment}}{}
Returns a PCSet of the literal compliment of the given object.

\end{fulllineitems}

\index{mpartner (core.PPCSetBase attribute)}

\begin{fulllineitems}
\phantomsection\label{_templates/core:core.PPCSetBase.mpartner}\pysigline{\bfcode{mpartner}}{}
Return a PCSet for the M-partner of the given object.

\end{fulllineitems}

\index{pcint (core.PPCSetBase attribute)}

\begin{fulllineitems}
\phantomsection\label{_templates/core:core.PPCSetBase.pcint}\pysigline{\bfcode{pcint}}{}
Returns the integer representation of a given object in prime form.

\end{fulllineitems}

\index{prime (core.PPCSetBase attribute)}

\begin{fulllineitems}
\phantomsection\label{_templates/core:core.PPCSetBase.prime}\pysigline{\bfcode{prime}}{}
Return a PCSet that represents the given object in prime form, taking
into account its canonical TTO's (set these with .canon(T, I, M)).

\end{fulllineitems}

\index{prime\_operation (core.PPCSetBase attribute)}

\begin{fulllineitems}
\phantomsection\label{_templates/core:core.PPCSetBase.prime_operation}\pysigline{\bfcode{prime\_operation}}{}
A property that returns (n, m) to perform on the given object via TnMm
in order to obtain its prime form.

\end{fulllineitems}

\index{setint (core.PPCSetBase attribute)}

\begin{fulllineitems}
\phantomsection\label{_templates/core:core.PPCSetBase.setint}\pysigline{\bfcode{setint}}{}
Returns the integer representation for the unique PC's in a given
object

\end{fulllineitems}

\index{subprimes() (core.PPCSetBase method)}

\begin{fulllineitems}
\phantomsection\label{_templates/core:core.PPCSetBase.subprimes}\pysiglinewithargsret{\bfcode{subprimes}}{\emph{limit=0}}{}
Yields the subsets of the given object which have a unique set-class.
Takes an optional limit argument with the same behavior as subsets().

\end{fulllineitems}

\index{subsets() (core.PPCSetBase method)}

\begin{fulllineitems}
\phantomsection\label{_templates/core:core.PPCSetBase.subsets}\pysiglinewithargsret{\bfcode{subsets}}{\emph{limit=0}}{}
Yields the subsets of the given object. Takes an optional argument,
which limits the subsets to those with a cardinality \textgreater{}= the limit.
With no argument, returns all subsets.

\end{fulllineitems}

\index{superprimes() (core.PPCSetBase method)}

\begin{fulllineitems}
\phantomsection\label{_templates/core:core.PPCSetBase.superprimes}\pysiglinewithargsret{\bfcode{superprimes}}{\emph{limit=0}}{}
Yields the supersets of the given object which have a unique set-class.
Takes an optional limit argument with the same behavior as supersets()

\end{fulllineitems}

\index{supersets() (core.PPCSetBase method)}

\begin{fulllineitems}
\phantomsection\label{_templates/core:core.PPCSetBase.supersets}\pysiglinewithargsret{\bfcode{supersets}}{\emph{limit=0}}{}
Yields the supersets of the given object. Takes an optional argument,
which limits the supersets to those with a cardinality \textless{}= the limit.
With no argument, returns all supersets.

\end{fulllineitems}

\index{zpartner (core.PPCSetBase attribute)}

\begin{fulllineitems}
\phantomsection\label{_templates/core:core.PPCSetBase.zpartner}\pysigline{\bfcode{zpartner}}{}
Property that returns the Z-partner of the given object if it exists,
otherwise returns None.

\end{fulllineitems}


\end{fulllineitems}

\index{PSet (class in core)}

\begin{fulllineitems}
\phantomsection\label{_templates/core:core.PSet}\pysiglinewithargsret{\strong{class }\code{core.}\bfcode{PSet}}{\emph{*args}, \emph{**kwargs}}{}
A class for pitch sets, which adds pitch set only methods.

\end{fulllineitems}

\index{SetRowBase (class in core)}

\begin{fulllineitems}
\phantomsection\label{_templates/core:core.SetRowBase}\pysiglinewithargsret{\strong{class }\code{core.}\bfcode{SetRowBase}}{\emph{*args}, \emph{**kwargs}}{}
Base class for PC/pitch sets and tone rows
\index{all\_rotations (core.SetRowBase attribute)}

\begin{fulllineitems}
\phantomsection\label{_templates/core:core.SetRowBase.all_rotations}\pysigline{\bfcode{all\_rotations}}{}
Return a flat list of objects for each possible TTO of the given object

\end{fulllineitems}

\index{copy() (core.SetRowBase method)}

\begin{fulllineitems}
\phantomsection\label{_templates/core:core.SetRowBase.copy}\pysiglinewithargsret{\bfcode{copy}}{\emph{pitches=None}}{}
Use to copy a ToneRow/PSet/PCSet with all data attributes.

\end{fulllineitems}

\index{default\_m() (core.SetRowBase method)}

\begin{fulllineitems}
\phantomsection\label{_templates/core:core.SetRowBase.default_m}\pysiglinewithargsret{\bfcode{default\_m}}{\emph{new\_m=None}}{}
Takes one argument as the new default argument for M operations.
(The default for Mod 12 is 5)
Without an argument, returns the current default m.

\end{fulllineitems}

\index{each\_n() (core.SetRowBase method)}

\begin{fulllineitems}
\phantomsection\label{_templates/core:core.SetRowBase.each_n}\pysiglinewithargsret{\bfcode{each\_n}}{}{}
Yields a number for each possible member in the object considering its
modulus.
(An object with a modulus of 12 would return {[}0, 1, 2...11{]})

\end{fulllineitems}

\index{each\_tto() (core.SetRowBase method)}

\begin{fulllineitems}
\phantomsection\label{_templates/core:core.SetRowBase.each_tto}\pysiglinewithargsret{\bfcode{each\_tto}}{}{}
Yields an (n, m) pair for each TTO that can be performed on the given
object

\end{fulllineitems}

\index{i() (core.SetRowBase method)}

\begin{fulllineitems}
\phantomsection\label{_templates/core:core.SetRowBase.i}\pysiglinewithargsret{\bfcode{i}}{\emph{sub\_n=0}}{}
Invert the object in place. If an argument is provided, also transpose
the object in place by that amount.

\end{fulllineitems}

\index{i\_rotations (core.SetRowBase attribute)}

\begin{fulllineitems}
\phantomsection\label{_templates/core:core.SetRowBase.i_rotations}\pysigline{\bfcode{i\_rotations}}{}
Returns a list of objects for each possible transposition of the given
object after inversion.

\end{fulllineitems}

\index{m\_rotations (core.SetRowBase attribute)}

\begin{fulllineitems}
\phantomsection\label{_templates/core:core.SetRowBase.m_rotations}\pysigline{\bfcode{m\_rotations}}{}
Returns a list of objects for each possible transposition of the given
object after M.

\end{fulllineitems}

\index{mi\_rotations (core.SetRowBase attribute)}

\begin{fulllineitems}
\phantomsection\label{_templates/core:core.SetRowBase.mi_rotations}\pysigline{\bfcode{mi\_rotations}}{}
Returns a list of objects for each possible transposition of the given
object after MI.

\end{fulllineitems}

\index{mod() (core.SetRowBase method)}

\begin{fulllineitems}
\phantomsection\label{_templates/core:core.SetRowBase.mod}\pysiglinewithargsret{\bfcode{mod}}{\emph{new\_mod=None}}{}
Takes one argument as the new modulus of the system.
Without an argument, returns the current modulus.

\end{fulllineitems}

\index{multiset() (core.SetRowBase method)}

\begin{fulllineitems}
\phantomsection\label{_templates/core:core.SetRowBase.multiset}\pysiglinewithargsret{\bfcode{multiset}}{\emph{value=None}}{}
Takes one boolean argument and determines if the object is a multiset.
(The default for all objects is False. ToneRows cannot be multisets)
Without an argument, returns the current setting.

\end{fulllineitems}

\index{ordered() (core.SetRowBase method)}

\begin{fulllineitems}
\phantomsection\label{_templates/core:core.SetRowBase.ordered}\pysiglinewithargsret{\bfcode{ordered}}{\emph{value=None}}{}
Takes one boolean argument and determines if the object is ordered.
(The default for PCSets is False. The default for PSets is True.)
Without an argument, returns the current setting.

\end{fulllineitems}

\index{pcs (core.SetRowBase attribute)}

\begin{fulllineitems}
\phantomsection\label{_templates/core:core.SetRowBase.pcs}\pysigline{\bfcode{pcs}}{}
Returns the pitch classes of the current set/row

\end{fulllineitems}

\index{ppc (core.SetRowBase attribute)}

\begin{fulllineitems}
\phantomsection\label{_templates/core:core.SetRowBase.ppc}\pysigline{\bfcode{ppc}}{}
Returns the pitches or pcs of a ToneRow, PCSet, or PSet taking into
account the ordered and multiset settings.

\end{fulllineitems}

\index{t() (core.SetRowBase method)}

\begin{fulllineitems}
\phantomsection\label{_templates/core:core.SetRowBase.t}\pysiglinewithargsret{\bfcode{t}}{\emph{sub\_n}}{}
Transpose the object in place by the argument provided.

\end{fulllineitems}

\index{t\_rotations (core.SetRowBase attribute)}

\begin{fulllineitems}
\phantomsection\label{_templates/core:core.SetRowBase.t_rotations}\pysigline{\bfcode{t\_rotations}}{}
Returns a list of objects for each possible transposition of the given
object.

\end{fulllineitems}

\index{uo\_pcs (core.SetRowBase attribute)}

\begin{fulllineitems}
\phantomsection\label{_templates/core:core.SetRowBase.uo_pcs}\pysigline{\bfcode{uo\_pcs}}{}
Returns unordered pitch classes in ascending order

\end{fulllineitems}

\index{uo\_pitches (core.SetRowBase attribute)}

\begin{fulllineitems}
\phantomsection\label{_templates/core:core.SetRowBase.uo_pitches}\pysigline{\bfcode{uo\_pitches}}{}
Returns the unordered pitches in ascending order

\end{fulllineitems}


\end{fulllineitems}

\index{invert() (in module core)}

\begin{fulllineitems}
\phantomsection\label{_templates/core:core.invert}\pysiglinewithargsret{\code{core.}\bfcode{invert}}{\emph{pitches}, \emph{sub\_n=0}}{}
Given an object and n, returns an object of the same type after TnI.
If n is not provided, n is assumed to be 0

\end{fulllineitems}

\index{multiply() (in module core)}

\begin{fulllineitems}
\phantomsection\label{_templates/core:core.multiply}\pysiglinewithargsret{\code{core.}\bfcode{multiply}}{\emph{pitches}, \emph{sub\_m}}{}
Given an object and n, returns an object of the same type after TnMm,
where m is required. (For mod 12, m is usually 5.)

\end{fulllineitems}

\index{transpose() (in module core)}

\begin{fulllineitems}
\phantomsection\label{_templates/core:core.transpose}\pysiglinewithargsret{\code{core.}\bfcode{transpose}}{\emph{pitches}, \emph{sub\_n=0}}{}
Given an object and n, returns an object of the same type after Tn

\end{fulllineitems}

\index{transpose\_multiply() (in module core)}

\begin{fulllineitems}
\phantomsection\label{_templates/core:core.transpose_multiply}\pysiglinewithargsret{\code{core.}\bfcode{transpose\_multiply}}{\emph{pitches}, \emph{sub\_n}, \emph{sub\_m}}{}
Given an object, n, and m, returns an object of the same type after TnMm.
All arguments are required.

\end{fulllineitems}

\index{transpose() (in module core)}

\begin{fulllineitems}
\pysiglinewithargsret{\code{core.}\bfcode{transpose}}{\emph{pitches}, \emph{sub\_n=0}}{}
Given an object and n, returns an object of the same type after Tn

\end{fulllineitems}

\index{invert() (in module core)}

\begin{fulllineitems}
\pysiglinewithargsret{\code{core.}\bfcode{invert}}{\emph{pitches}, \emph{sub\_n=0}}{}
Given an object and n, returns an object of the same type after TnI.
If n is not provided, n is assumed to be 0

\end{fulllineitems}

\index{multiply() (in module core)}

\begin{fulllineitems}
\pysiglinewithargsret{\code{core.}\bfcode{multiply}}{\emph{pitches}, \emph{sub\_m}}{}
Given an object and n, returns an object of the same type after TnMm,
where m is required. (For mod 12, m is usually 5.)

\end{fulllineitems}

\index{transpose\_multiply() (in module core)}

\begin{fulllineitems}
\pysiglinewithargsret{\code{core.}\bfcode{transpose\_multiply}}{\emph{pitches}, \emph{sub\_n}, \emph{sub\_m}}{}
Given an object, n, and m, returns an object of the same type after TnMm.
All arguments are required.

\end{fulllineitems}



\subsection{Data Inspection}
\label{_templates/data_inspection:data-inspection}\label{_templates/data_inspection::doc}\label{_templates/data_inspection:id1}

\subsection{Operators}
\label{_templates/operators:operators}\label{_templates/operators::doc}\label{_templates/operators:id1}
Pitches or pitch classes can be added to an existing set with the += or = + idioms.
The addition operator returns a new object, so it can also be used to instantiate a new objet.
For examle:
\textgreater{}\textgreater{}\textgreater{} a += {[}3, 9{]}
\textgreater{}\textgreater{}\textgreater{} b = b + {[}0{]}
\textgreater{}\textgreater{}\textgreater{} c = a + b

Addition can also be used for evaluation such as:
\textgreater{}\textgreater{}\textgreater{} a = PSet(0, 1, 11)
\textgreater{}\textgreater{}\textgreater{} a + {[}3, 9{]} == {[}0, 1, 3, 9, 11{]}
\textgreater{}\textgreater{}\textgreater{} Out: True

For information on the == and != evaluation, refer to :ref:'evaluation'

To return new set or row instances modified by a TTO, import and use the following:

\begin{Verbatim}[commandchars=\\\{\}]
\PYG{g+gp}{\textgreater{}\textgreater{}\textgreater{} }\PYG{k+kn}{from} \PYG{n+nn}{sator.core} \PYG{k+kn}{import} \PYG{n}{transpose}\PYG{p}{,} \PYG{n}{invert}\PYG{p}{,} \PYG{n}{multiply}\PYG{p}{,} \PYG{n}{transpose\PYGZus{}multiply}
\end{Verbatim}


\subsection{TTO's}
\label{_templates/ttos:ttos}\label{_templates/ttos::doc}\label{_templates/ttos:tto-s}
TTO is an acronym for ``Twelve Tone Operators.''


\subsection{Properties}
\label{_templates/properties::doc}\label{_templates/properties:properties}\label{_templates/properties:id1}

\subsection{Tone Rows}
\label{_templates/tone_rows:id1}\label{_templates/tone_rows::doc}\label{_templates/tone_rows:tone-rows}

\chapter{Indices and tables}
\label{index:indices-and-tables}\begin{itemize}
\item {} 
\emph{genindex}

\item {} 
\emph{modindex}

\item {} 
\emph{search}

\end{itemize}


\renewcommand{\indexname}{Python Module Index}
\begin{theindex}
\def\bigletter#1{{\Large\sffamily#1}\nopagebreak\vspace{1mm}}
\bigletter{c}
\item {\texttt{core}}, \pageref{_templates/core:module-core}
\end{theindex}

\renewcommand{\indexname}{Index}
\printindex
\end{document}
